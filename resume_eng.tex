\documentclass[10pt, letterpaper]{article}

\usepackage[
    ignoreheadfoot, 
    top=2 cm, 
    bottom=2 cm, 
    left=2 cm, 
    right=2 cm, 
    footskip=1.0 cm, 
    ]{geometry} 
\usepackage{titlesec} 
\usepackage{tabularx} 
\usepackage{array} 
\usepackage[dvipsnames]{xcolor} 
\definecolor{primaryColor}{RGB}{0, 79, 144} 
\usepackage{enumitem} 
\usepackage{fontawesome5} 
\usepackage{amsmath, amsfonts, amssymb, amsthm} 
\usepackage[
    pdftitle={Luciano Melodia Curriculum Vitae},
    pdfauthor={Luciano Melodia},
    pdfcreator={Luciano Melodia},
    colorlinks=false,
    hidelinks
]{hyperref} 
\usepackage[pscoord]{eso-pic} 
\usepackage{calc} 
\usepackage{bookmark} 
\usepackage{lastpage} 
\usepackage{changepage} 
\usepackage{paracol} 
\usepackage{ifthen} 
\usepackage{needspace} 
\usepackage{iftex} 

\ifPDFTeX
    \input{glyphtounicode}
    \pdfgentounicode=1
    
    \usepackage[utf8]{inputenc}
    \usepackage{lmodern}
\fi



\AtBeginEnvironment{adjustwidth}{\partopsep0pt} 
\pagestyle{empty} 
\setcounter{secnumdepth}{0} 
\setlength{\parindent}{0pt} 
\setlength{\topskip}{0pt} 
\setlength{\columnsep}{0cm} 
\makeatletter
\let\ps@customFooterStyle\ps@plain 
\patchcmd{\ps@customFooterStyle}{\thepage}{
    \color{gray}\textit{\small Luciano Melodia -- Curriculum Vitae \thepage{} of \pageref*{LastPage}}
}{}{} 
\makeatother
\pagestyle{customFooterStyle}

\titleformat{\section}{\needspace{4\baselineskip}\bfseries\large}{}{0pt}{}[\vspace{1pt}\titlerule]

\titlespacing{\section}{-1pt}{0.3 cm}{0.2 cm} 

\renewcommand\labelitemi{$\circ$} 
\newenvironment{highlights}{
    \begin{itemize}[
        topsep=0 cm,
        parsep=0 cm,
        partopsep=0pt,
        itemsep=0pt,
        leftmargin=0.4 cm + 10pt
    ]
}{
    \end{itemize}
} 

\newenvironment{highlightsforbulletentries}{
    \begin{itemize}[
        topsep=0 cm,
        parsep=0 cm,
        partopsep=0pt,
        itemsep=0pt,
        leftmargin=10pt
    ]
}{
    \end{itemize}
} 


\newenvironment{onecolentry}{
    \begin{adjustwidth}{
        0.2 cm + 0.00001 cm
    }{
        0.2 cm + 0.00001 cm
    }
}{
    \end{adjustwidth}
} 

\newenvironment{twocolentry}[2][]{
    \onecolentry
    \def\secondColumn{#2}
    \setcolumnwidth{\fill, 4.5 cm}
    \begin{paracol}{2}
}{
    \switchcolumn \raggedleft \secondColumn
    \end{paracol}
    \endonecolentry
} 

\newenvironment{header}{
    \setlength{\topsep}{0pt}\par\kern\topsep\centering\linespread{1.5}
}{
    \par\kern\topsep
} 

\newcommand{\placelastupdatedtext}{
  \AddToShipoutPictureFG*{
    \put(
        \LenToUnit{\paperwidth-2 cm-0.2 cm+0.05cm},
        \LenToUnit{\paperheight-1.0 cm}
    ){\vtop{{\null}\makebox[0pt][c]{
        \small\color{gray}\textit{\today}
    }}}
  }
}

\let\hrefWithoutArrow\href

\renewcommand{\href}[2]{\hrefWithoutArrow{#1}{\ifthenelse{\equal{#2}{}}{ }{#2 }\raisebox{.15ex}{\footnotesize \faExternalLink*}}}


\begin{document}
\newcommand{\AND}{\unskip
	\cleaders\copy\ANDbox\hskip\wd\ANDbox
	\ignorespaces
}
\newsavebox\ANDbox
\sbox\ANDbox{}

\placelastupdatedtext
\begin{header}
	\textbf{\fontsize{24 pt}{24 pt}\selectfont Luciano Melodia}

	\vspace{0.3 cm}

	\normalsize
	\mbox{{\color{black}\footnotesize\faMapMarker*}\hspace*{0.13cm}Heckenweg 23, 91056 Erlangen}
	\kern 0.25 cm
	\AND
	\kern 0.25 cm
	\mbox{\hrefWithoutArrow{mailto:luciano.melodia@fau.de}{\color{black}{\footnotesize\faEnvelope[regular]}\hspace*{0.13cm}luciano.melodia@fau.de}}
	\kern 0.25 cm
	\AND
	\kern 0.25 cm
	\mbox{\hrefWithoutArrow{tel:+49 175 3372526}{\color{black}{\footnotesize\faPhone*}\hspace*{0.13cm}+49 175 3372526}}
\end{header}

\vspace{0.3 cm - 0.3 cm}

\section{Education}
\begin{twocolentry}{
		\textit{Oct. 2024 – Mar. 2026}} \textbf{Master of Science Mathematics} \\
		\textit{Friedrich-Alexander University Erlangen–Nuremberg}\\
		\textit{Minor: Digital Humanities}
\end{twocolentry}
\begin{onecolentry}
	\begin{highlights}
		\item \textbf{Thesis:} Universal Coefficients and the Mayer--Vietoris Sequence for Groupoid Homology
	\end{highlights}
\end{onecolentry}

\begin{twocolentry}{
		\textit{Oct. 2021 – Sept. 2024}}
	\textbf{Bachelor of Science Mathematics} \\
	\textit{Friedrich-Alexander University Erlangen–Nuremberg} \\
	\textit{Minor: Computer Science}
\end{twocolentry}
\begin{onecolentry}
	\begin{highlights}
		\item \textbf{Thesis:} Algebraic and Topological Persistence
	\end{highlights}
\end{onecolentry}

\begin{twocolentry}{
		\textit{Apr. 2015 – Mar. 2018}}
	\textbf{Master of Arts Information Science} \\
	\textit{University of Regensburg}\\
	\textit{Minor: Digital Humanities}
\end{twocolentry}
\begin{onecolentry}
	\begin{highlights}
		\item \textbf{Thesis:} Deep Learning for Estimating Absorbed Radiation Dose in Nuclear Medicine Diagnostics
	\end{highlights}
\end{onecolentry}

\begin{twocolentry}{
		\textit{Oct. 2012 – Mar. 2015}}
	\textbf{Bachelor of Arts German Philology} \\
	\textit{University of Regensburg} \\
	\textit{Majors: German and Italian Philology, Information Science, Media Informatics}
\end{twocolentry}
\begin{onecolentry}
	\begin{highlights}
		\item \textbf{Thesis:} Development of a Punctuation Platform with Linguistic Modules for Information Retrieval
	\end{highlights}
\end{onecolentry}

\begin{twocolentry}{
		\textit{Oct. 2012 – Apr. 2013}}
	\textbf{University of Regensburg} \\
	\textit{Concurrent IT Training Program}
\end{twocolentry}

\section{Professional Experience}
\begin{twocolentry}{
		\textit{Erlangen}

		\textit{Apr. 2023 – Sept. 2026}}
	\textbf{FAU Department of Mathematics}

	\textit{Research Assistant}
\end{twocolentry}

\vspace{0.1 cm}
\begin{onecolentry}
  \begin{highlights}
    \item Tutor for Topology (2023, 2026), Functional Analysis (2026), Real Analysis~III (2024, 2026), Real Analysis~II (2025), Linear Algebra~I (2024), Mathematics for Engineers~A2: Calculus (2025), Mathematics for Engineers~A4: Stochastics (2025), and Topology and Applications (2024).
    \item Substitute lecturer for courses taught by Prof.\ Dr.\ Li and Prof.\ Dr.\ Meusburger on the Tietze extension theorem, connectedness and path-connectedness, and continuity.
    \item Lecturer for course units on mathematical proof writing.
    \item Exam proctoring, grading, and mentoring.
  \end{highlights}
\end{onecolentry}
\vspace{0.1 cm}

\begin{twocolentry}{
		\textit{Erlangen}

		\textit{Aug. 2021 – Aug. 2022}}
	\textbf{Corscience GmbH \& Co. KG}

	\textit{Working Student}
\end{twocolentry}

\vspace{0.10 cm}
\begin{onecolentry}
	\begin{highlights}
	  \item Deep convolutional networks on multi-GPU systems for automatic detection of calibration spikes in ECG data achieving >99\,\% accuracy (10-fold cross-validation, approx.\ 1M real-world samples, state of the art).
	  \item Residual networks for detection of ECG traces in documents with an IoU of approx.\ 98\,\% (10-fold cross-validation, approx.\ 10M augmented samples), state-of-the-art image segmentation.
	  \item Image segmentation of ECG traces using matrix factorization methods with an IoU of approx.\ 99\,\%, statistically highly significant, state of the art.
	\end{highlights}
\end{onecolentry}
\vspace{0.1 cm}

\begin{twocolentry}{
		\textit{Erlangen}

		\textit{Sept. 2018 – Dec. 2021}}
	\textbf{Siemens Energy AG}

	\textit{Research Associate}
\end{twocolentry}

\vspace{0.1 cm}
\begin{onecolentry}
	\begin{highlights}
	  \item Development of a novel topology-based interpolation method for industrial sensor data and handwriting data; published at IWCIA, with an open-source implementation.
	  \item Design of a method to estimate the capacity of neural networks for signal data; reduction of training costs by approx.\ €25{,}000 per year; presented at ICPR, source code available as open source.
	  \item Development of a hierarchical KKS classifier for power-plant sensors using topological data analysis (Betti curves) with accuracies of up to 93\,\%; published at PKDD, with an open-source implementation.
	  \item Extensive teaching and examination experience in data science and computer science (lectures, seminars, tutorials, e-exams) with excellent course evaluations.
	  \item Supervision of multiple B.Sc.\ and M.Sc.\ theses on sensor-signal classification, ECG analysis, gas-sensor time series, and industrial dashboards.
	  \item Technologies: Python (3.8/3.9), TensorFlow~2.x, CUDA~11, cuDNN~8, Linux (Ubuntu, Arch), and Windows.
	  \item Teaching assistant for Conceptual Modeling (2019), New Technologies in Data Management (2019, 2020, 2021), Process-Oriented Information Systems (2019, 2020, 2021), Topological Data Analysis (2020), Homological Data Analysis (2021).
	  \item Lecturer for Knowledge Discovery in Databases (2021).
	\end{highlights}
\end{onecolentry}
\vspace{0.1 cm}

\begin{twocolentry}{
	\textit{Regensburg}

	\textit{June 2015 – Mar. 2018}}
	\textbf{mb Support GmbH}

	\textit{Scientific Stuff}
\end{twocolentry}

\vspace{0.10 cm}
\begin{onecolentry}
	\begin{highlights}
		\item Industrial document digitization pipeline using high-performance OCR.
		\item Integration of the Asterisk telephony API into Openviva C2.
		\item Market and statistical analysis using deep learning.
	\end{highlights}
\end{onecolentry}
\vspace{0.1 cm}

\begin{twocolentry}{
		\textit{University of Regensburg}

		\textit{Oct. 2013 – Sept. 2015}}
	\textbf{Chair of German Linguistics}

	\textit{Research Assistant}
\end{twocolentry}

\vspace{0.10 cm}
\begin{onecolentry}
	\begin{highlights}
	  \item Subject-matter editing and proofreading of academic texts and examinations.
	  \item Organization and coordination of academic conferences.
	  \item Technical maintenance and updates of the university website.
	  \item Design and implementation of an academic social network.
	\end{highlights}
\end{onecolentry}
\vspace{0.1 cm}

\begin{twocolentry}{
	\textit{Regensburg}

	\textit{Sept. 2012 – Dec. 2015}}
	\textbf{Apostelkeller}

	\textit{Chef}
\end{twocolentry}

\vspace{0.1 cm}
\begin{onecolentry}
	\begin{highlights}
	  \item Planning and preparing menus for up to 140 guests.
	  \item Front-of-house support and waiting tables with direct guest interaction.
	  \item Inventory management and stock control in the kitchen.
	\end{highlights}
\end{onecolentry}


\vspace{0.1 cm}
\begin{twocolentry}{
		\textit{Regensburg}

		\textit{Oct. 2012 – Aug. 2014}}
	\textbf{Anatol GmbH \& Co. KG}

	\textit{Translator}
\end{twocolentry}

\vspace{0.10 cm}
\begin{onecolentry}
	\begin{highlights}
		\item Translation between Italian, German, Polish, and English.
	\end{highlights}
\end{onecolentry}

\section{Skills}
\begin{onecolentry}
  \textbf{Programming:} Python (expert), Rust (advanced), C++ (basic)
\end{onecolentry}
\begin{onecolentry}
  \textbf{Web Technologies:} HTML5, CSS3 (expert), JavaScript, PHP (advanced)
\end{onecolentry}
\begin{onecolentry}
  \textbf{Typesetting:} \LaTeX{} (expert)
\end{onecolentry}
\begin{onecolentry}
  \textbf{Operating Systems:} Linux (Arch, Ubuntu) and macOS (expert), Windows (advanced)
\end{onecolentry}
\begin{onecolentry}
  \textbf{Languages:} German (native), English (C2), Italian (C2), Polish (B2), Spanish (A2)
\end{onecolentry}
\begin{onecolentry}
  \textbf{Sports:} Lifting (240kg Deadlift, 120kg Bench, 150kg Squat), Muay Thai 4:2:1 – W:L:D $\bullet$, Weng Chun {\color{Green}$\bullet$}
\end{onecolentry}
\begin{onecolentry}
  \textbf{Hobbies:} Competitive cooking, fiction (Amor Towles \& Benedict Jacka)
\end{onecolentry}

\section{Publications}
\begin{samepage}
	\begin{twocolentry}{2021}
		\textbf{Homological Time Series Analysis of Sensor Signals from Power Plants.}

		\mbox{\textbf{\textit{Luciano Melodia}}}, \mbox{Richard Lenz}
	\end{twocolentry}
	\begin{onecolentry}
		\href{https://doi.org/10.1007/978-3-030-93736-2\_22}{10.1007/978-3-030-93736-2\_22}
	\end{onecolentry}
	\begin{twocolentry}{2021}
		\textbf{Estimate of the Neural Network Dimension Using Algebraic Topology and Lie Theory.}

		\mbox{\textbf{\textit{Luciano Melodia}}}, \mbox{Richard Lenz}
	\end{twocolentry}
	\begin{onecolentry}
		\href{https://doi.org/10.1007/978-3-030-68821-9\_2}{10.1007/978-3-030-68821-9\_2}
	\end{onecolentry}
	\begin{twocolentry}{2020}
		\textbf{Persistent Homology as a Stopping Criterion for Voronoi Interpolation.}

		\mbox{\textbf{\textit{Luciano Melodia}}}, \mbox{Richard Lenz}
	\end{twocolentry}
	\begin{onecolentry}
		\href{https://doi.org/10.1007/978-3-030-51002-2\_3}{10.1007/978-3-030-51002-2\_3}
	\end{onecolentry}
	\begin{twocolentry}{2015}
		\textbf{On the Use of the Verb \textit{brauchen} with and without \textit{zu} with the Infinitive.}

		\mbox{\textbf{\textit{Luciano Melodia}}}
	\end{twocolentry}
	\begin{onecolentry}
		\href{https://www.logos-verlag.de/cgi-bin/engbuchmid?isbn=3808&lng=eng&id=}{ISBN 978-3-8325-3808-8}
	\end{onecolentry}
\end{samepage}

\section{Conferences}
\begin{onecolentry}
    \textbf{Reviewer:} \href{https://logconference.org/program-committee/}{Learning on Graphs (LOG, 2022–24)}, \href{https://www.iaria.org/conferences/DBKDA.html}{DBKDA (2020–24)}, \href{https://gt-rl.github.io/}{GT-RL @ ICLR (2022)}, \href{https://tda-in-ml.github.io/committee}{TDA in ML @ NeurIPS (2020)}.
\end{onecolentry}
\begin{onecolentry}
	\textbf{Author:} \href{https://link.springer.com/conference/icpr}{ICPR (2021)}, \href{https://iwcia2020.wordpress.com/}{IWCIA (2020)}, \href{https://ecmlpkdd2019.org/}{ECML PKDD (2019–2020)}, \href{https://www.uni-regensburg.de/sprache-literatur-kultur/germanistik/internationales/forschung-und-lehre/index.html}{Destandardization and Standard Variety (2013)}.
\end{onecolentry}
\begin{onecolentry}
	\textbf{Invited Talk:} \href{https://cnc.ac.in/assets/uploads/Faculty/profile/Mr.\%20Rajeshkumar\%20Resume\_.pdf}{International Conference on Practical Mathematical Discourse (2020)}.
\end{onecolentry}
\begin{onecolentry}
	\textbf{Attendance:} \href{https://www.math.fau.de/lie-gruppen/}{Colloquium on Lie Groups}, \href{https://sigmod2020.org/}{SIGMOD/PODS (2020)}, \href{https://www.uni-regensburg.de/sprache-literatur-kultur/germanistik/internationales/forschung-und-lehre/index.html}{Language Management and Orthography (2015)}.
\end{onecolentry}

\section{Awards, Scholarships, and Service}
\begin{twocolentry}{\textit{2024}}
  \href{https://logconference.org/program-committee/}{Best Reviewer Award, Learning on Graphs (LOG)}
\end{twocolentry}
\begin{twocolentry}{\textit{2024}}
  \href{https://www.stmwk.bayern.de/studenten/foerderung-und-stipendien/forster-stipendium.html}{Recipient of the Oskar Karl Forster Scholarship}
\end{twocolentry}
\begin{twocolentry}{\textit{2024}}
  Student Representative for the Department of Mathematics, Friedrich-Alexander University Erlangen–Nuremberg
\end{twocolentry}
\begin{twocolentry}{\textit{2019–2020}}
  \href{https://gi.de/}{Member of the German Informatics Society}
\end{twocolentry}
\begin{twocolentry}{\textit{2017–2018}}
  \href{https://www.uni-regensburg.de/informatik-data-science/maschinelles-lernen-insb-uncertainty-quantification/startseite/index.html}{Member of the Computational Intelligence and Machine Learning Research Group (CIML)}
\end{twocolentry}
\begin{twocolentry}{\textit{2016}}
  Student Representative for the Faculty of Language, Literature, and Cultural Studies, University of Regensburg
\end{twocolentry}
\end{document}