\documentclass[10pt, letterpaper]{article}

\usepackage[
    ignoreheadfoot, 
    top=2 cm, 
    bottom=2 cm, 
    left=2 cm, 
    right=2 cm, 
    footskip=1.0 cm, 
    ]{geometry} 
\usepackage{titlesec} 
\usepackage{tabularx} 
\usepackage{array} 
\usepackage[dvipsnames]{xcolor} 
\definecolor{primaryColor}{RGB}{0, 79, 144} 
\usepackage{enumitem} 
\usepackage{fontawesome5} 
\usepackage{amsmath, amsfonts, amssymb, amsthm} 
\usepackage[
    pdftitle={Luciano Melodias Lebenslauf},
    pdfauthor={Luciano Melodia},
    pdfcreator={Luciano Melodia},
    colorlinks=false,
    hidelinks
]{hyperref} 
\usepackage[pscoord]{eso-pic} 
\usepackage{calc} 
\usepackage{bookmark} 
\usepackage{lastpage} 
\usepackage{changepage} 
\usepackage{paracol} 
\usepackage{ifthen} 
\usepackage{needspace} 
\usepackage{iftex} 

\ifPDFTeX
    \input{glyphtounicode}
    \pdfgentounicode=1
    
    \usepackage[utf8]{inputenc}
    \usepackage{lmodern}
\fi



\AtBeginEnvironment{adjustwidth}{\partopsep0pt} 
\pagestyle{empty} 
\setcounter{secnumdepth}{0} 
\setlength{\parindent}{0pt} 
\setlength{\topskip}{0pt} 
\setlength{\columnsep}{0cm} 
\makeatletter
\let\ps@customFooterStyle\ps@plain 
\patchcmd{\ps@customFooterStyle}{\thepage}{
    \color{gray}\textit{\small Luciano Melodia -- Lebenslauf \thepage{} von \pageref*{LastPage}}
}{}{} 
\makeatother
\pagestyle{customFooterStyle}

\titleformat{\section}{\needspace{4\baselineskip}\bfseries\large}{}{0pt}{}[\vspace{1pt}\titlerule]

\titlespacing{\section}{-1pt}{0.3 cm}{0.2 cm} 

\renewcommand\labelitemi{$\circ$} 
\newenvironment{highlights}{
    \begin{itemize}[
        topsep=0 cm,
        parsep=0 cm,
        partopsep=0pt,
        itemsep=0pt,
        leftmargin=0.4 cm + 10pt
    ]
}{
    \end{itemize}
} 

\newenvironment{highlightsforbulletentries}{
    \begin{itemize}[
        topsep=0 cm,
        parsep=0 cm,
        partopsep=0pt,
        itemsep=0pt,
        leftmargin=10pt
    ]
}{
    \end{itemize}
} 


\newenvironment{onecolentry}{
    \begin{adjustwidth}{
        0.2 cm + 0.00001 cm
    }{
        0.2 cm + 0.00001 cm
    }
}{
    \end{adjustwidth}
} 

\newenvironment{twocolentry}[2][]{
    \onecolentry
    \def\secondColumn{#2}
    \setcolumnwidth{\fill, 5 cm}
    \begin{paracol}{2}
}{
    \switchcolumn \raggedleft \secondColumn
    \end{paracol}
    \endonecolentry
} 

\newenvironment{header}{
    \setlength{\topsep}{0pt}\par\kern\topsep\centering\linespread{1.5}
}{
    \par\kern\topsep
} 

\newcommand{\placelastupdatedtext}{
  \AddToShipoutPictureFG*{
    \put(
        \LenToUnit{\paperwidth-2 cm-0.2 cm+0.05cm},
        \LenToUnit{\paperheight-1.0 cm}
    ){\vtop{{\null}\makebox[0pt][c]{
        \small\color{gray}\textit{\today}
    }}}
  }
}

\let\hrefWithoutArrow\href

\renewcommand{\href}[2]{\hrefWithoutArrow{#1}{\ifthenelse{\equal{#2}{}}{ }{#2 }\raisebox{.15ex}{\footnotesize \faExternalLink*}}}


\begin{document}
\newcommand{\AND}{\unskip
	\cleaders\copy\ANDbox\hskip\wd\ANDbox
	\ignorespaces
}
\newsavebox\ANDbox
\sbox\ANDbox{}

\placelastupdatedtext
\begin{header}
	\textbf{\fontsize{24 pt}{24 pt}\selectfont Luciano Melodia}

	\vspace{0.3 cm}

	\normalsize
	\mbox{{\color{black}\footnotesize\faMapMarker*}\hspace*{0.13cm}Heckenweg 23, 91056 Erlangen}
	\kern 0.25 cm
	\AND
	\kern 0.25 cm
	\mbox{\hrefWithoutArrow{mailto:luciano.melodia@fau.de}{\color{black}{\footnotesize\faEnvelope[regular]}\hspace*{0.13cm}luciano.melodia@fau.de}}
	\kern 0.25 cm
	\AND
	\kern 0.25 cm
	\mbox{\hrefWithoutArrow{tel:+49 175 3372526}{\color{black}{\footnotesize\faPhone*}\hspace*{0.13cm}+49 175 3372526}}
\end{header}

\vspace{0.3 cm - 0.3 cm}

\section{Ausbildung}
\begin{twocolentry}{
		\textit{Okt. 2024 – März 2026}} \textbf{Master of Science Mathematik} \\
		\textit{Friedrich-Alexander-Universität Erlangen–Nürnberg}\\
		\textit{Nebenfach: Digitale Geisteswissenschaften}
\end{twocolentry}
\begin{onecolentry}
	\begin{highlights}
		\item \textbf{Thesis:} Universelle Koeffizienten und Mayer-Vietoris Sequenz für Moore Homologie
	\end{highlights}
\end{onecolentry}

\begin{twocolentry}{
		\textit{Okt. 2021 – Sept. 2024}}
	\textbf{Bachelor of Science Mathematik} \\
	\textit{Friedrich-Alexander-Universität Erlangen–Nürnberg}\\
	\textit{Nebenfach: Informatik}
\end{twocolentry}
\begin{onecolentry}
	\begin{highlights}
		\item \textbf{Thesis:} Algebraische und topologische Persistenz
	\end{highlights}
\end{onecolentry}

\begin{twocolentry}{
		\textit{April 2015 – März 2018}}
	\textbf{Master of Arts Informationswissenschaft} \\
	\textit{Universität Regensburg}\\
	\textit{Nebenfach: Digitale Geisteswissenschaften}
\end{twocolentry}
\begin{onecolentry}
	\begin{highlights}
		\item \textbf{Thesis:} Deep Learning zur Schätzung der absorbierten Strahlungsdosis in der nuklearmedizinischen Diagnostik
	\end{highlights}
\end{onecolentry}

\begin{twocolentry}{
		\textit{Okt. 2012 – März 2015}}
	\textbf{Bachelor of Arts Deutsche Philologie} \\
	\textit{Universität Regensburg}\\
	\textit{Hauptfächer: Deutsch, Italienisch, Informationswissenschaft, Medieninformatik}
\end{twocolentry}
\begin{onecolentry}
	\begin{highlights}
		\item \textbf{Thesis:} Entwicklung einer Interpunktionsplattform mit linguistischem Information Retrieval
	\end{highlights}
\end{onecolentry}

\begin{twocolentry}{
	\textit{Okt. 2012 – April 2013}}
	\textbf{Universität Regensburg} \\
	\textit{Studienbegleitende IT-Ausbildung}
\end{twocolentry}


\section{Berufserfahrung}
\begin{twocolentry}{
		\textit{Erlangen}

		\textit{April 2023 – Sept. 2026}}
	\textbf{FAU Department Mathematik}

	\textit{Wissenschaftliche Hilfskraft}
\end{twocolentry}

\vspace{0.1 cm}
\begin{onecolentry}
  \begin{highlights}
    \item Tutor in Topologie (2023, 2026), Funktionalanalysis (2026), Analysis~3 (2024, 2026), Analysis~2 (2025), Lineare Algebra~1 (2024), Mathematik für Ingenieure~A2: Analysis (2025), Mathematik für Ingenieure~A4: Stochastik (2025), sowie Topology and Applications (2024).
    \item Vertretung von Vorlesungen (Prof.\ Dr.\ Li, Prof.\ Dr.\ Meusburger) zu: \\
    	  Tietzeschem Fortsetzungssatz, Zusammenhang und Wegzusammenhang, Stetigkeit.
    \item Dozent für Veranstaltungseinheiten zum mathematischen Beweisen.
    \item Aufsicht, Korrektur und Betreuung schriftlicher Prüfungen.
  \end{highlights}
\end{onecolentry}
\vspace{0.1 cm}

\begin{twocolentry}{
		\textit{Erlangen}

		\textit{Aug. 2021 – Aug. 2022}}
	\textbf{Corscience GmbH \& Co. KG}

	\textit{Werkstudent}
\end{twocolentry}

\vspace{0.10 cm}
\begin{onecolentry}
	\begin{highlights}
	  \item Tiefe konvolutionale Netze auf Multi-GPU-Systemen zur automatischen Erkennung von Kalibrierspitzen in EKG-Daten; >99\,\% Genauigkeit (10-fache Kreuzvalidierung, ca.\ 1 Mio.\ Realbeispiele, State of the Art).
	  \item Residualnetze zur Detektion von EKG-Kurven in Dokumenten; IoU von ca.\ 98\,\% (10-fache Kreuzvalidierung, ca.\ 10 Mio.\ augmentierte Beispiele), State-of-the-Art-Bildsegmentierung.
	  \item Bildsegmentierung von EKG-Kurven mittels Matrixfaktorisierungsverfahren; IoU von ca.\ 99\,\%, statistisch hochsignifikant, State of the Art.
	\end{highlights}
\end{onecolentry}
\vspace{0.1 cm}

\begin{twocolentry}{
		\textit{Erlangen}

		\textit{Sept. 2018 – Dez. 2021}}
	\textbf{Siemens Energy AG}

	\textit{Wissenschaftlicher Mitarbeiter}
\end{twocolentry}

\vspace{0.1 cm}
\begin{onecolentry}
	\begin{highlights}
	  \item Entwicklung eines neuartigen topologiebasierten Interpolationsverfahrens für industrielle Sensordaten und Handschriftdaten; veröffentlicht auf der IWCIA, mit Open-Source-Implementierung.
	  \item Konzeption einer Methode zur Abschätzung der Kapazität neuronaler Netze für Signaldaten; Reduktion der Trainingskosten um jährlich ca.\ 25.000\,€; präsentiert auf der ICPR, Quellcode als Open Source verfügbar.
	  \item Aufbau eines hierarchischen KKS-Klassifikators für Kraftwerkssensoren unter Verwendung topologischer Datenanalyse (Betti-Kurven) mit Genauigkeiten von bis zu 93\,\%; veröffentlicht auf der PKDD, mit Open-Source-Implementierung.
	  \item Umfangreiche Lehr- und Prüfungserfahrung in Data Science und Informatik (Vorlesungen, Seminare, Übungen, E-Klausuren) mit exzellenten Veranstaltungsevaluationen.
	  \item Betreuung mehrerer BSc-/MSc-Abschlussarbeiten zu Sensorsignalklassifikation, EKG-Analyse, Gassensor-Zeitreihen und industriellen Dashboards.
	  \item Technologie: Python (3.8/3.9), TensorFlow~2.x, CUDA~11, cuDNN~8, Linux (Ubuntu, Arch) und Windows.
	  \item Übungsleiter in Konzeptionelle Modellierung (2019), Neue Technologien im Datenmanagement (2019, 2020, 2021), Prozessorientierte Informationssysteme (2019, 2020, 2021), Topologische Datenanalyse (2020), Homologische Datenanalyse (2021).
	  \item Dozent in Knowledge Discovery in Databases (2021).
	\end{highlights}
\end{onecolentry}
\vspace{0.1 cm}

\begin{twocolentry}{
	\textit{Regensburg}

	\textit{Juni 2015 – März 2018}}
	\textbf{mb Support GmbH}

	\textit{Wissenschaftlicher Mitarbeiter}
\end{twocolentry}

\vspace{0.10 cm}
\begin{onecolentry}
	\begin{highlights}
		\item Industrielle Dokumenten-Digitalisierungspipeline mit Hochleistungs-OCR.
		\item Asterisk-Telefonie-API-Integration in Openviva C2.
		\item Markt- und Statistik-Analyse mit Deep Learning.
	\end{highlights}
\end{onecolentry}
\vspace{0.1 cm}

\begin{twocolentry}{
		\textit{Universität Regensburg}

		\textit{Okt. 2013 – Sept. 2015}}
	\textbf{Lehrstuhl für Deutsche Sprachwissenschaft}

	\textit{Wissenschaftliche Hilfskraft}
\end{twocolentry}

\vspace{0.10 cm}
\begin{onecolentry}
	\begin{highlights}
	  \item Fachlektorat und Korrektur wissenschaftlicher Texte und Prüfungen.
	  \item Organisation und Koordination wissenschaftlicher Konferenzen.
	  \item Technische Betreuung und Aktualisierung der Universitätswebseite.
	  \item Konzeption und Implementierung eines wissenschaftlichen sozialen Netzwerks.
	\end{highlights}
\end{onecolentry}
\vspace{0.1 cm}

\begin{twocolentry}{
	\textit{Regensburg}

	\textit{Sept. 2012 – Dez. 2015}}
	\textbf{Apostelkeller}

	\textit{Koch}
\end{twocolentry}

\vspace{0.1 cm}
\begin{onecolentry}
	\begin{highlights}
	  \item Planung und Zubereitung von Menüs für bis zu 140 Gäste.
	  \item Service- und Kellnertätigkeiten im direkten Gästekontakt.
	  \item Lager- und Warenbestandsmanagement in der Küche.
	\end{highlights}
\end{onecolentry}


\vspace{0.1 cm}
\begin{twocolentry}{
		\textit{Regensburg}

		\textit{Okt. 2012 – Aug. 2014}}
	\textbf{Anatol GmbH \& Co. KG}

	\textit{Übersetzer}
\end{twocolentry}

\vspace{0.10 cm}
\begin{onecolentry}
	\begin{highlights}
		\item Übersetzungen zwischen Italienisch -- Deutsch -- Polnisch -- Englisch.
	\end{highlights}
\end{onecolentry}

\section{Fähigkeiten}
\begin{onecolentry}
  \textbf{Programmierung:} Python (Expertenkenntnisse), Rust (gute Kenntnisse), C
++ (gute Kenntnisse)
\end{onecolentry}
\begin{onecolentry}
  \textbf{Webtechnologien:} HTML5, CSS3 (Expertenkenntnisse), JavaScript, PHP (sehr gute Kenntnisse)
\end{onecolentry}
\begin{onecolentry}
  \textbf{Textsatz:} \LaTeX{} (Expertenkenntnisse)
\end{onecolentry}
\begin{onecolentry}
  \textbf{Betriebssysteme:} Linux (Arch, Ubuntu) und macOS (Expertenkenntnisse), Windows (gute Kenntnisse)
\end{onecolentry}
\begin{onecolentry}
  \textbf{Sprachen:} Deutsch (Muttersprache), Englisch (C2), Italienisch (C2), Polnisch (B2), Spanisch (A2)
\end{onecolentry}
\begin{onecolentry}
  \textbf{Sport:} Dreikampf (240kg Kreuz, 120kg Bank, 150kg Beuge), Muay Thai (4:2:1) $\bullet$, Kung Fu {\color{Green}$\bullet$}
\end{onecolentry}
\begin{onecolentry}
  \textbf{Hobbys:} Kochen (kompetitiv), Belletristik (Amor Towles \& Benedict Jacka)
\end{onecolentry}

\section{Publikationen}
\begin{samepage}
	\begin{twocolentry}{2021}
		\textbf{Homological Time Series Analysis of Sensor Signals from Power Plants.}

		\mbox{\textbf{\textit{Luciano Melodia}}}, \mbox{Richard Lenz}
	\end{twocolentry}
	\begin{onecolentry}
		\href{https://doi.org/10.1007/978-3-030-93736-2\_22}{10.1007/978-3-030-93736-2\_22}
	\end{onecolentry}
	\begin{twocolentry}{2021}
		\textbf{Estimate of the Neural Network Dimension Using Algebraic Topology and Lie Theory.}

		\mbox{\textbf{\textit{Luciano Melodia}}}, \mbox{Richard Lenz}
	\end{twocolentry}
	\begin{onecolentry}
		\href{https://doi.org/10.1007/978-3-030-68821-9\_2}{10.1007/978-3-030-68821-9\_2}
	\end{onecolentry}
	\begin{twocolentry}{2020}
		\textbf{Persistent Homology as a Stopping Criterion for Voronoi Interpolation.}

		\mbox{\textbf{\textit{Luciano Melodia}}}, \mbox{Richard Lenz}
	\end{twocolentry}
	\begin{onecolentry}
		\href{https://doi.org/10.1007/978-3-030-51002-2\_3}{10.1007/978-3-030-51002-2\_3}
	\end{onecolentry}
	\begin{twocolentry}{2015}
		\textbf{Zur Verwendung des Paradigmas \textit{brauchen} mit und ohne \textit{zu} mit Infinitiv.}

		\mbox{\textbf{\textit{Luciano Melodia}}}
	\end{twocolentry}
	\begin{onecolentry}
		\href{https://www.logos-verlag.de/cgi-bin/engbuchmid?isbn=3808&lng=eng&id=}{ISBN 978-3-8325-3808-8}
	\end{onecolentry}
\end{samepage}

\section{Konferenzen}
\begin{onecolentry}
    \textbf{Gutachter:} \href{https://logconference.org/program-committee/}{Learning on Graphs (LOG, 2022–24)}, \href{https://www.iaria.org/conferences/DBKDA.html}{DBKDA (2020–24)}, \href{https://gt-rl.github.io/}{GT-RL @ ICLR (2022)}, \href{https://tda-in-ml.github.io/committee}{TDA in ML @ NeurIPS (2020)}.
\end{onecolentry}
\begin{onecolentry}
	\textbf{Autor:} \href{https://link.springer.com/conference/icpr}{ICPR (2021)}, \href{https://iwcia2020.wordpress.com/}{IWCIA (2020)}, \href{https://ecmlpkdd2019.org/}{ECML PKDD (2019–2020)}, \href{https://www.uni-regensburg.de/sprache-literatur-kultur/germanistik/internationales/forschung-und-lehre/index.html}{Destandardisierung und Standardvarietät (2013)}.
\end{onecolentry}
\begin{onecolentry}
	\textbf{Gastvortrag:} \href{https://cnc.ac.in/assets/uploads/Faculty/profile/Mr.\%20Rajeshkumar\%20Resume\_.pdf}{International Conference on Practical Mathematical Discourse (2020)}.
\end{onecolentry}
\begin{onecolentry}
	\textbf{Teilnahme:} \href{https://www.math.fau.de/lie-gruppen/}{Kolloquium zu Lie-Gruppen}, \href{https://sigmod2020.org/}{SIGMOD/PODS (2020)}, \href{https://www.uni-regensburg.de/sprache-literatur-kultur/germanistik/internationales/forschung-und-lehre/index.html}{Sprachmanagement und Orthografie (2015)}.
\end{onecolentry}

\section{Auszeichnungen, Stipendien und Gremientätigkeit}
\begin{twocolentry}{\textit{2024}}
  \href{https://logconference.org/program-committee/}{Auszeichnung als bester Gutachter der Konferenz Learning on Graphs (LOG)}
\end{twocolentry}
\begin{twocolentry}{\textit{2024}}
  \href{https://www.stmwk.bayern.de/studenten/foerderung-und-stipendien/forster-stipendium.html}{Stipendiat des Oskar-Karl-Forster-Stipendiums}
\end{twocolentry}
\begin{twocolentry}{\textit{2024}}
  Studierendenvertreter für das Department Mathematik der Friedrich-Alexander-Universität Erlangen–Nürnberg
\end{twocolentry}
\begin{twocolentry}{\textit{2019–2020}}
  \href{https://gi.de/}{Mitglied der Gesellschaft für Informatik e.\,V.}
\end{twocolentry}
\begin{twocolentry}{\textit{2017–2018}}
  \href{https://www.uni-regensburg.de/informatik-data-science/maschinelles-lernen-insb-uncertainty-quantification/startseite/index.html}{Mitglied der AG Computational Intelligence and Machine Learning (CIML)}
\end{twocolentry}
\begin{twocolentry}{\textit{2016}}
  Studierendenvertreter für die Fakultät für Sprach-, Literatur- und Kulturwissenschaften der Universität Regensburg
\end{twocolentry}
\end{document}